\section*{Exercice 190 -- Correcteur PI}
\setcounter{exo}{0}

\textit{Exercices de Marc Derumaux}

On considère une FTBO de fonction de transfert $\text{FTBO}(p)= \dfrac{2}{p^2+2p+1}$. L'asservissement (en
boucle fermée) n'étant pas précis, on adopte une correction PI (proportionnelle intégrale) de la
forme $C( p)=K\dfrac{1+\tau p}{p}$.

\subparagraph{}
\textit{Montrer que le correcteur choisi revient à un correcteur PI du type $K_P + \dfrac{K_I}{p}$ 
et identifier $K_P$ et $K_I$ en fonction de $K$ et $\tau$.
}
\ifprof
\begin{corrige}
$C(p)=K\dfrac{1+\tau p}{p} = K\tau + \dfrac{K}{p}$ d'où $K_P = K\tau$ et $K_I = K$.
\end{corrige}
\else
\fi


\subparagraph{}
\textit{Tracer le diagramme de Bode du correcteur seul en précisant les caractéristiques en fonction de
$K$ et $\tau$. }
\ifprof
\begin{corrige}
\end{corrige}
\else
\fi

\subparagraph{}
\textit{Tracer le diagramme de Bode asymptotique ainsi que l'allure du diagramme de Bode réel de la
FTBO. }
\ifprof
\begin{corrige}
\end{corrige}
\else
\fi




\subparagraph{}
\textit{Pour éviter de dégrader les marges de stabilité, la pulsation de cassure du correcteur est placée une
décade avant la pulsation propre de la FTBO. En déduire la constante de temps $\tau$.}
\ifprof
\begin{corrige}
La pulsation de cassure du correcteur est $\dfrac{1}{\tau}$. Elle est placée une décade avant $\omega_0$
pour $\tau=\SI{10}{s}$.
\end{corrige}
\else
\fi

\subparagraph{}
\textit{On souhaite une bande passante de \SI{3}{rad/s}. Déterminer la valeur de $K$ assurant cette bande passante
et en déduire les marges de gain et de phase.}
\ifprof
\begin{corrige}
Pour $K=1$, le gain de la FTBO en $\omega=\SI{3rad}{s}$ vaut $\left| \dfrac{1+\tau j \omega }{j \omega } \times \dfrac{2}{1+2j\omega + \left(j\omega\right)^2}\right|=2$. Il
faut donc $K=0,5$ pour que la FTBO soit unitaire en $\omega =\SI{3}{rad/s}$. La marge de phase vaut
alors -145\degres (argument de la FTBO en $\omega=\SI{3}{rad/s}$ ), soit 35\degres de marge de phase. La marge
de gain est infinie car la phase ne coupe jamais la valeur -180\degres.
\end{corrige}
\else
\fi