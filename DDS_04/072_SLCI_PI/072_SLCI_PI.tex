\section*{Exercice 146 -- Correcteur PI}
\setcounter{exo}{0}
%CCS PSI 2005



La consigne de la régulation de l’inclinaison $\Psi(t)$ du châssis par rapport à la verticale
est notée $\Psi_C(t)$. On introduit un correcteur de fonction de transfert $C(p)$
qui élabore le signal $w(t)$ (de transformée de Laplace $W(p)$) à partir de l’écart
$\varepsilon(t)=\Psi_C(t)-\Psi(t)$.


\subparagraph{}
\textit{Compléter le schéma bloc de l’asservissement en faisant apparaître la régulation de l’inclinaison. }
\ifprof
\begin{corrige}
\end{corrige}
\else
\fi


\begin{center}
\includegraphics[width=\linewidth]{975_01}%
\end{center}


La régulation d’inclinaison du Segway® consiste à maintenir la consigne $\Psi_C(t)$
nulle. Cette régulation est réalisée si, quelle que soit l’inclinaison $\alpha(t)$
du conducteur, la sortie $\Psi(t)$ converge vers $\Psi_C(t)$, valeur nulle ici.

Le conducteur agit directement sur la valeur de $\alpha(t)$ pour accélérer ou décélérer.
Pour le système Segway®, conducteur exclu, le paramètre $\alpha(t)$  peut être considéré
comme une perturbation.
Un correcteur proportionnel $C(p)=K_C$ est envisagé.


\subparagraph{}
\textit{Calculer l’inclinaison $\Psi(t)$ du châssis en régime permanent, lorsque la
perturbation  $\alpha(t)$ est un échelon d’amplitude $\alpha_0$. Le cahier des charges est-il
satisfait ?}
\ifprof
\begin{corrige}
\end{corrige}
\else
\fi

Un correcteur proportionnel intégral $C(p)=K_i\left( 1 +\dfrac{1}{T_i p} \right)$ est envisagé.

\subparagraph{}
\textit{Démontrer que ce correcteur permet de satisfaire le cahier des charges vis-à-vis de l’écart en régime permanent pour une perturbation en échelon.}
\ifprof
\begin{corrige}
\end{corrige}
\else
\fi

On souhaite dimensionner le correcteur. Pour cela, on étudie le schéma-bloc
construit précédemment et on considère alors $\alpha(t)=0$. La Fonction de Transfert en
Boucle Ouverte est pour cet asservissement : $\text{FTBO}(p)=C(p)F_2(p)$.



\subparagraph{}
\textit{Tracer les diagrammes de Bode asymptotiques et réels (allure uniquement)
de la fonction de transfert $F_2(p)$ et tracer les diagrammes de Bode asymptotiques
de la fonction de transfert du correcteur $C(p)$, en utilisant les
paramètres $K_i$ et $T_i$. Préciser les valeurs caractéristiques sur les diagrammes.}
\ifprof
\begin{corrige}
\end{corrige}
\else
\fi

On impose $\omega_i = \dfrac{1}{T_i}=\dfrac{\omega_c}{10}$ où $\omega_c$ est la pulsation de coupure à de la
FTBO corrigée par le correcteur proportionnel intégral.

\subparagraph{}
\textit{Déterminer $\omega_c$ telle que la marge de la FTBO soit $M_{\varphi}=45\degres$. En
déduire la valeur de $T_i$.}
\ifprof
\begin{corrige}
\end{corrige}
\else
\fi



\subparagraph{}
\textit{Déterminer alors $K_i$ tel que $\omega_c$ soit effectivement la pulsation de coupure
à \SI{0}{dB} de la FTBO corrigée.}
\ifprof
\begin{corrige}
\end{corrige}
\else
\fi





\begin{center}
%\includegraphics[width=\linewidth]{976_02}%
\end{center}

%
%
%
%
%
%\begin{obj}
%Vérifier les performances de l’asservissement d’inclinaison par rapport à la verticale.
%\end{obj}
%
%Pour une utilisation confortable et sûre, le Segway doit satisfaire les performances
%énoncées dans le tableau extrait du cahier des charges.
%\begin{center}
%\includegraphics[width=\linewidth]{976_01}%
%\end{center}
%La régulation d’inclinaison du Segway® est réalisée par :
%\begin{itemize}
%\item un moto-réducteur qui permet de délivrer un couple $C_m(t)=K_m u(t)$ où $u(t)$ est une grandeur de commande et $K_m=\SI{24}{N.m.V^{-1}}$;
%\item le système mécanique dont les équations, dans le cas où l’angle $\alpha(t)$ n’est pas supposé constant, se met
%sous la forme :
%$\left\{
%\begin{array}{l}
%\dot{V}(t)=\dfrac{1}{D}\left( B\ddot{\chi}(t)+2\dfrac{C_m(t)}{R}\right) \\
%\left(DA-B^2\right)\ddot{\chi}(t)=2\left(\dfrac{B}{R}+D\right)C_m(t)+DC\chi(t)
%\end{array}
%\right.
%$ 
%avec
%$\left\{
%\begin{array}{l}
%A=\SI{90}{kg.m^2} \\
%B=\SI{75}{kg.m} \\
%C=\SI{750}{kg.m^2.s^{-2}} \\
%D=\SI{125}{kg} \\
%R=\SI{240}{mm} \\
%\chi(t)=\alpha(t)+\psi(t) 
%\end{array}
%\right.
%$.
%\end{itemize}
%
%
%Par commodité de signe, la notation est utilisée dans les équations ci-dessus. Les conditions initiales sont toutes nulles.
%
%
%\subparagraph{}
%\textit{}
%\ifprof
%\begin{corrige}
%\end{corrige}
%\else
%\fi
%
%\begin{center}
%%\includegraphics[width=\linewidth]{976_02}%
%\end{center}
%
%
%\subparagraph{}
%\textit{Analyser la stabilité du système d’entrée $u(t)$ et de sortie $\psi(t)$ en étudiant la fonction de transfert $F_1(p)=\dfrac{\Psi(p)}{U(p)}$.}% Pouvait-on s'attendre à ce résultat ?}
%\ifprof
%\begin{corrige}
%\end{corrige}
%\else
%\fi
%
%
%On note alors $H_1(p)=\dfrac{K_1}{\dfrac{p^2}{\omega_1^2} - 1}$. 
%
%Les valeurs numériques utilisées par la suite seront : $\omega_1=\SI{4,1}{rad.s^{-1}}$ et
%$K_S = K_mK_1 = \SI{0,24}{rad.V^{-1}}$.
%
%Afin de stabiliser le système, la grandeur de commande $U(p)$ est élaborée à partir des mesures de $\dot{\Psi}$ (réalisée par le gyromètre), et de ${\Psi}$( (réalisée par combinaison de la mesure du gyromètre et du pendule). Le schéma-blocs obtenu est celui du document réponse.
%
%
%\subparagraph{}
%\textit{Dans le cas où $\alpha = 0$, déterminer, en fonction de $K_S$, $k_p$, $k_v$ et $\omega_1$ la
%fonction de transfert $F_2(p)=\dfrac{\Psi(p)}{W(p)}$.}
%\ifprof
%\begin{corrige}
%\end{corrige}
%\else
%\fi
%
%
%
%\subparagraph{}
%\textit{Déterminer les conditions sur $k_v$ et sur $k_p$ pour que le système soit stable.}
%\ifprof
%\begin{corrige}
%\end{corrige}
%\else
%\fi
%
%$F_2(p)$ est une fonction de transfert du second ordre pouvant se mettre sous la forme :
%$F_2(p)=\dfrac{\Psi(p)}{W(p)}=\dfrac{K_2}{1+\dfrac{2\xi}{\omega_0}p+\dfrac{p^2}{\omega_0^2}}$.
%
%
%\subparagraph{}
%\textit{Déterminer les expressions de $K_2$, $\xi$ et $\omega_0$.}
%\ifprof
%\begin{corrige}
%\end{corrige}
%\else
%\fi
%
%On choisit une pulsation propre proche de celle du système mécanique, c’est
%à dire $\omega_0 = 1,5\omega_1=\SI{6,15}{rad.s^{-1}}$.
%
%
%
%\subparagraph{}
%\textit{Déterminer les valeurs de $k_p$ et de $k_v$ telles que le temps de réponse à 5\% soit minimal.}
%\ifprof
%\begin{corrige}
%\end{corrige}
%\else
%\fi
%
%
%
%%\subparagraph{}
%%\textit{}
%%\ifprof
%%\begin{corrige}
%%\end{corrige}
%%\else
%%\fi
%
%
\begin{enumerate}
\item ...
\item ...
\item $F_2(p)=\dfrac{K_S}{\dfrac{p^2}{\omega_1^2}+pk_v K_S +k_pK_S - 1}$.
\item $k_p>\dfrac{1}{K_S}$ et $k_v>0$.
\item $K_2 = \dfrac{K_S}{k_p K_S - 1}$, $\omega_0 = \omega_1\sqrt{k_p K_S - 1}$ et $\xi = \dfrac{1}{2}\dfrac{k_vK_S\omega_1}{\sqrt{k_p K_S - 1}}$.
\item $k_v = \SI{2,15}{rad.s^{-1}s}$, $K_2 \simeq \SI{0,1}{rad.V^{-1}}$, $k_p \simeq \SI{13,54}{V.rad^{-1}}$.
\end{enumerate}